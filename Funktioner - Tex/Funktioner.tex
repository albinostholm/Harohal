% ----------------------------------------------------------
% Paket och definitioner
\documentclass[11pt, titlepage, oneside, a4paper]{article}	% Dokumentspecifikation
\usepackage[swedish]{babel}									% Språk
\usepackage[T1]{fontenc}									% Teckenkodning för output
\usepackage[utf8]{inputenc}									% Teckenkodning för input
\usepackage{amssymb, graphicx, fancyhdr}					% Symboler och grafik
\usepackage{amsmath}										% Matematiska uttryck
\usepackage{amsthm,algorithm,algorithmic,yhmath,enumitem,lscape} % Algoritmer
\usepackage{float}											% Placeringar av grafik mm.
\usepackage{hyperref, url}									% Webbadresser och referenser
\usepackage{listings}										% Lista källkod
\lstset{
	frame=single,
	title=\lstname,
	numbers=left,
	basicstyle=\tiny
}															% Formatering för listning av källkod

\def\school{Erik Dahlbergsgymnasiet, Jönköping}				% Skolans namn
\def\class{T4}												% FYLL I DIN KLASS
\def\programme{Teknikprogrammet, \class}					% Program
\def\documenttype{Laborationsrapport}						% Typ av dokument
\def\course{Webbutveckling\\ Webbserverprogramering\\Datalagring}	% FYLL I KURS ELLER OMRÅDE
\def\title{Harohal - Funktioner}											% FYLL I UPPG. NAMN
\def\student{Harohal \& co}										% FYLL I DITT NAMN
\def\email{-}												% FYLL I DIN E-POST
\def\graders{Rickard Karlsson}								% FYLL I NAMN PÅ LÄRARE

\setlength{\parindent}{0pt}									% Ta bort indentering
\setlength{\parskip}{10pt}									% Lägg till mellanrum vid nytt stycke

% ----------------------------------------------------------
% Dokumentstart
\begin{document}

% ----------------------------------------------------------
% Framsida
\begin{titlepage}
	\thispagestyle{empty}
	\begin{normalsize}
		\begin{tabular}{@{}p{\textwidth}@{}}
			\textbf{\school \hfill \today} \\
			\textbf{\programme \hfill \documenttype} \\
		\end{tabular}
	\end{normalsize}
	\vspace{25mm}
	\begin{center}
		\huge{\textbf{\course}}\\
		\vspace{10mm}
		\LARGE{\title} \\
		\vspace{15mm}
        \LARGE{version 1.0} \\
        \vspace{10mm}
		\begin{large}
			\begin{tabular}{ll}
				\textbf{Namn} & \student \\
				\textbf{E-post} & \email \\
                
			\end{tabular}
		\end{large}
		\vfill
        \vfill
		\large{\textbf{Bedömning}}\\
		\mbox{\large{\graders}}
	\end{center}
\end{titlepage}
    
% Sidhuvud
\lhead{\footnotesize\student, \today}
\rhead{\nouppercase{\footnotesize\leftmark}}
\pagestyle{fancy}
\renewcommand{\headrulewidth}{0.2pt} 

\newpage

\section*{Funktioner}

\begin{tabular}{ll}
\hline
Funktionsnamn & Lägg till Annons                        \\ \hline
Beskrivning   & Lägga till en ny Annons på startsidan \\ \hline
Indata        & namn, beskrivning, länk, bild         \\ \hline
Utdata        & annonsID                              \\ \hline
\end{tabular}

\begin{tabular}{ll}
\hline
Funktionsnamn & Ta bort Annons                        \\ \hline
Beskrivning   & Ta bort en Annons på startsidan \\ \hline
Indata        & annonsID         \\ \hline
Utdata        & Aktiv = falsk                              \\ \hline
\end{tabular}

\begin{tabular}{ll}
\hline
Funktionsnamn & Ändra Annons                       \\ \hline
Beskrivning   & Ändra beskrivning och titel av en annons \\ \hline
Indata        & annonsID, namn, beskrivning, datum, bild, länk   \\ \hline
Utdata        & Lyckad redigering                             \\ \hline
\end{tabular}

\begin{tabular}{ll}
\hline
Funktionsnamn & Skapa konto                     \\ \hline
Beskrivning   & Skapar ett nytt konto \\ \hline
Indata        & Förnamn, efternamn, mail, personnummer, lösenord, telefon   \\ \hline
Utdata        & personID              \\ \hline
\end{tabular}

\begin{tabular}{ll}
\hline
Funktionsnamn & Logga in                     \\ \hline
Beskrivning   & Logga in på webbsidan \\ \hline
Indata        & Lösenord, användarnamn   \\ \hline
Utdata        & personID              \\ \hline
\end{tabular}

\begin{tabular}{ll}
\hline
Funktionsnamn & Byta lösenord                    \\ \hline
Beskrivning   & Byta lösenord på webbsidan \\ \hline
Indata        & Gammalt lösenord, Nytt lösenord   \\ \hline
Utdata        & True/False              \\ \hline
\end{tabular}

\begin{tabular}{ll}
\hline
Funktionsnamn & Verifiera konto                    \\ \hline
Beskrivning   & Verifierar ditt konto från email  \\ \hline
Indata        & GUID   \\ \hline
Utdata        & bekräftatEmail = sant, mail, förnamn, efternamn    \\ \hline
\end{tabular}

\begin{tabular}{ll}
\hline
Funktionsnamn & Ny nyhet                    \\ \hline
Beskrivning   & För att lägga till nyhet \\ \hline
Indata        & nyhetsdatum, nyhetsbeskrivning, publiceradatum, titel \\ \hline
Utdata        & True/false, nyhets ID              \\ \hline
\end{tabular}

\begin{tabular}{ll}
\hline
Funktionsnamn & Ta bort nyhet                   \\ \hline
Beskrivning   & Ta bort en nyhet ifrån sida \\ \hline
Indata        & nyhetsdatum, nyhetsbeskrivning, publiceradatum, titel \\ \hline
Utdata        & Publicerad = falsk            \\ \hline
\end{tabular}

\begin{tabular}{ll}
\hline
Funktionsnamn & Ändra nyhet                   \\ \hline
Beskrivning   & Ändra beskrivning och titel av en nyhet \\ \hline
Indata        & nyhetsdatum, nyhetsbeskrivning, publiceradatum, titel, nyhetsID \\ \hline
Utdata        & True/False            \\ \hline
\end{tabular}

\begin{tabular}{ll}
\hline
Funktionsnamn & Boka                  \\ \hline
Beskrivning   & Bokar en massage med en massör under en viss tid \\ \hline
Indata        & anställdID, tjänstID, startTid, slutTid pris, \\ \hline
Utdata        & True/false, ev.Ny rad på ordrar med unikt orderID, ev.stadie = aktiv            \\ \hline
\end{tabular}

\begin{tabular}{ll}
\hline
Funktionsnamn & Avboka obetald order                  \\ \hline
Beskrivning   & Avbokning för en bokning som är obetald \\ \hline
Indata        & OrderID \\ \hline
Utdata        & True/False, ev.Lyckad borttagning     \\ \hline
\end{tabular}

\begin{tabular}{ll}
\hline
Funktionsnamn & Avboka betald order                  \\ \hline
Beskrivning   & Avbokning för en bokning som är betald \\ \hline
Indata        & OrderID \\ \hline
Utdata        & True/False, ev.Lyckad borttagning, särskild administrativ återbetalning  \\ \hline
\end{tabular}

\begin{tabular}{ll}
\hline
Funktionsnamn & Hämta anställdinfo                  \\ \hline
Beskrivning   & Hämtar information kring en viss anställd \\ \hline
Indata        & anställdID \\ \hline
Utdata        & Namn, beskrivning, bild, behandlarMän/Kvinnor  \\ \hline
\end{tabular}

\begin{tabular}{ll}
\hline
Funktionsnamn & Lägg till anställd                 \\ \hline
Beskrivning   & Lägger till en ny anställd i systemet \\ \hline
Indata        & namn, beskrivning, tjänstID, behandlarMän/Kvinnor, bild \\ \hline
Utdata        & anställdID  \\ \hline
\end{tabular}

\begin{tabular}{ll}
\hline
Funktionsnamn & Ta bort anställd                 \\ \hline
Beskrivning   & Inaktiverar en anställd i systemet \\ \hline
Indata        & anställdID   \\ \hline
Utdata        & Aktiv = falsk  \\ \hline
\end{tabular}

\begin{tabular}{ll}
\hline
Funktionsnamn & Hämta schema                 \\ \hline
Beskrivning   & Hämtar en anställds schema för en specifik vecka \\ \hline
Indata        & anställdID, vecka, år \\ \hline
Utdata        & anställdID, starttid, sluttid, datum  \\ \hline
\end{tabular}

\begin{tabular}{ll}
\hline
Funktionsnamn & Ta bort schema                 \\ \hline
Beskrivning   & Tar bort en anställdstid för ett pass \\ \hline
Indata        & anställdID, starttid, vecka, år \\ \hline
Utdata        & True/False, ev lyckad borttagning  \\ \hline
\end{tabular}

\begin{tabular}{ll}
\hline
Funktionsnamn & Kopira schematid                 \\ \hline
Beskrivning   & Kopierar schema ifrån en annan veckas schema \\ \hline
Indata        & Sluttider, starttider, vecka, nyvecka \\ \hline
Utdata        & True/False, ev nytt schema  \\ \hline
\end{tabular}

\begin{tabular}{ll}
\hline
Funktionsnamn & Ny tjänst              \\ \hline
Beskrivning   & Lägger till en ny tjänst i systemet \\ \hline
Indata        & namn, beskrivning, tid, pris, bild \\ \hline
Utdata        & tjänstID  \\ \hline
\end{tabular}

\begin{tabular}{ll}
\hline
Funktionsnamn & Ta bort tjänst              \\ \hline
Beskrivning   & Flagga tjänst som inaktiv \\ \hline
Indata        & tjänstID  \\ \hline
Utdata        & Aktiv = falsk  \\ \hline
\end{tabular}

\begin{tabular}{ll}
\hline
Funktionsnamn & Ändra tjänst              \\ \hline
Beskrivning   & Ändrar informationen om tjänst \\ \hline
Indata        & tjänstID, namn, beskrivning, bild, pris, tid,   \\ \hline
Utdata        & True/False, Lyckad ändring  \\ \hline
\end{tabular}

\bibliography{references} 
\bibliographystyle{unsrt}
\addcontentsline{toc}{section}{Referenser}

\newpage
\appendix

\end{document}